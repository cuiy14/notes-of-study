\documentclass[UTF8,a4paper]{ctexart}
\usepackage{amsmath}
\usepackage[svgnames]{xcolor}
\usepackage{verbatim}
\usepackage{graphicx}
\usepackage{abstract}
\pagestyle{plain}
\title{\textbf{女性裸体艺术——古典艺术复兴中涌现的璀璨明珠}}
\date{}
\author{崔岩 \qquad \qquad 2014010785\\手机:18813119878\\邮箱:cuiy14@mails.tsinghua.edu.cn }

\begin{document}
\maketitle
\begin{abstract}
文艺复兴是欧洲历史上一段重要的时期。在这一时期,人文主义艺术家们利用包括女性裸体艺术在内的多样艺术形式来表现现世生活,引领大众的目光由天堂转移到日常生活,重新寻找到人的尊严与价值。本文以介绍文艺复兴时期女性裸体艺术的发展为切入点,较为全面地分析了女性裸体艺术在复兴古典艺术的过程中蓬勃发展的原因,同时也简要阐述了其在帮助大众发现``人的价值''的过程中所起到的关键作用。	
		
\end{abstract}
%\tableofcontents
%\section{文艺复兴中女性裸体艺术的繁荣}
%\keywords{女性裸体艺术\qquad 文艺复兴 \qquad 发展原因  \qquad 历史作用 }

\section{文艺复兴中女性裸体艺术的繁荣}
漫长黑暗的中世纪是欧罗巴土地上一段难熬的岁月——教皇的专制统治,教会的黑暗压迫,帝国王朝残酷无情的惩治······但是,在这表面的僵死顽固之下,在社会的某些不为统治阶级所熟知的角落里,却一直有着一条发源于古希腊罗马文化的人文小溪汩汩流淌着:这条并没有被战火掩埋,也没有被教会的腐朽统治填灭的溪水,从古希腊罗马的废墟中涌流而出,在历史的深隧中逐渐壮大,最终在14\~16世纪的意大利首先冲出重重封锁,重现天日,在广大的资产阶级市民阶层产生了深远的影响,开启了一股重寻``人的尊严''的新历程。

这是一段风起云涌、英才迭出的传奇时代,也是艺术不断更新、绽放异彩的时代。在这股经过几百年酝酿而终于喷薄而出的文化狂潮中,文艺复兴三杰首先接下历史的重任,扛起人文主义的大旗:在达·芬奇的笔下,俊美慈祥的“丽达圣母”用自己裸露的丰满的乳房为通体赤裸的耶稣哺乳;强悍的米开朗琪罗始终坚持着以桀骜不驯、不可挫败的姿态面对着教会对自己作品的恶意非难;风流潇洒的拉斐尔则用端庄秀丽、温柔多情、丰腴健美的圣母来给大众带来审美的享受。在他们的带领下,一批弄潮好手们相跃而出;他们也争相用他们自己的独特方式来描绘世界,传达广大新兴资产阶级市民对于人文思潮的渴望——乔尔乔尼的名作《沉睡的维纳斯》中,冰清玉洁的维纳斯在广袤明丽的田园上安详地沉睡着,爱神裸露的优美的胴体、明朗圣洁的心灵将人们卷入如诗如画的理想境界;“威尼斯的旗手”提香则更近一层,《浴后的维纳斯》中,一丝不挂、横躺在华丽居室中的爱神维纳斯用她那脉脉含情的双眸挑逗着观众,手中拿着的红玫瑰传达着她对于爱情的向往——这哪里还是那个高雅圣洁的神灵,明明就是现实中的一个鲜活的贵妇人!后来,西班牙的艺术大师鲁本斯在《美惠三女神》中描摹了三个赤裸的丰满妇女,在他的真诚兼粗暴的笔触下,优雅宁静的少女胴体没有了,取而代之的是更加放肆的、充满肉欲感的女性裸体:肥厚臃肿的躯体,充满视觉冲击力的近乎外溢的皮下脂肪,这种种看似粗俗的描摹实际上强烈表现了鲁本斯对于那个积聚酿造了这么多物质财富的资产阶级制度的由衷赞美;所谓的粗暴性,其实更应该算是一种浓墨重彩的挥毫!后来的偏好艳丽色彩与优雅装饰的“罗可可”文化则是将对裸体女性的描绘推上了巅峰:《浴后的狄安娜》将那富于青春、美艳绝伦的女性裸体呈现在广大观众眼前,晶莹剔透的女性裸体让人联想到“刚从缎子枕衾里爬出来的娇滴滴的贵族女性”\footnote{帅松林,《审美的历程》,P144,清华大学出版社,2014 }······

从文艺复兴三杰首开艺术创新历史先河,经过威尼斯画派和鲁本斯等艺术大家的发展,再经过后来的``罗可可''艺术时期的升华,女性裸体艺术终于由稚嫩走向了成熟,凝聚了大量杰出画家、雕刻家等的心血,成为了复兴古典艺术时期备受瞩目的艺术瑰宝!

\section{女性裸体艺术兴盛原因分析}
文艺复兴时期,以“复兴古典文化”为名的艺术家们,在这场轰轰烈烈、史无前例的对封建禁欲主义神学的大扫荡中,对于女性裸体艺术形式情有独钟,甚至偏爱过分。在历史的发展过程中,随着复兴活动的开展,他们对女性身体的描绘尺度越来越大,对女性胴体的风格表达,也经历了由含蓄宁静,到张扬挑逗,再到简单粗俗,直至最后有些放荡轻佻的意味。这到底意味着什么呢?为什么他们对女性裸体艺术偏爱有加呢?

一切历史现象的解释都要回到当初的历史背景,从社会中寻求根源。经过研究,我发现这是一个多维度的问题,其中有着复杂的因素,包括宗教层次,历史层次,社会层次,审美层次等多方面。下面,我将依次阐述自己对这个问题的看法,并不揣冒昧地请求阅者的批评指正。

\subsection{宗教层次}
法国著名美学家丹纳提出,“作品的产生取决于时代精神和周围的风俗,并强调时代精神和风俗“是艺术品最后的解释,也是决定一切的基本原因。” \footnote{丹纳,《艺术哲学》,北京人民美术出版社,1963}这一观点在文艺复兴时期的女性裸体艺术的兴盛中得到了充分的验证。

随着新航路的开辟、海外市场的不断扩大,原本陈旧的个体作坊之类的小关系已经不能满足资产阶级的需求了。此时他们所能做的只有彻底摧毁旧的生产关系。而生产关系决定上层建筑,与生产关系大变革相应的,必定是一场文化的大发展、大清洗。在当时占据主体地位的基督教文化中,肉体是与人的原罪相联系的,因而基督教认为人的肉体是肮脏的,人体外露是一种诲淫诲盗的行为,并由此引发了中世纪漫长而愚昧的“禁欲主义”。但是,文艺复兴时期的人们已经难以忍受教会专断独行的统治,不愿再接受苦行僧般的“禁欲”生活;为了打破这种宗教给大众带来的宿命般的束缚,或者明确的说,为了重新找回人的尊严,将大众的注意力由人造的玩偶——上帝——身上移开,而重新转移到现实生活,当时的艺术家们极力宣扬人文主义的思想。而女性裸体艺术,这一独特的艺术形式,似乎是天生针对教会的“肉体生来就是邪恶的”这一观点的,因而也得到了艺术家们的青睐。

现在,我们可以看到,文艺复兴时期的裸女主要有两种,一种是女神,一种是俗世女子。对裸体女神的描绘主要集中在文艺复兴前期,如《维纳斯的诞生》,《浴后的维纳斯》,《爱使马尔斯与维纳斯结合》等;这些作品都是取材于古希腊罗马神话,也就是所谓的“异教”文化;这些画家采用宗教题材,事实上并不是为了宣扬宗教教义,恰恰相反,他们深刻地意识到了“对于完全由宗教培育起来的群众感情来说,要掀起巨大的风暴,就必须让群众的切身利益披上宗教的外衣出现”\footnote{恩格斯,《路德维希·费尔巴哈和德国古典哲学的终结》,见《马克思恩格斯选集》卷四,p251,人民出版社}。也就是说,他们是借助宗教的噱头,却将其本质完全换成对美好现世生活的描绘,来使人们产生对宗教的厌恶和对现世的留恋。到了后期,这种宗教的外衣完全不必要了——假如不考虑图画的名字,我们完全分别不出拉斐尔的《椅中圣母》与我们常见的妩媚动人的青春女性有何区别!

可见,文艺复兴时期的巨匠们是把“裸体女性艺术作为反禁欲主义最重要的突破口”\footnote{帅松林,《审美的历程》,P128,清华大学出版社,2014}来对待的。正是借助着这一个突破口,他们才得以打开中世纪苦闷压抑的基督教在人们思想上打下的坚固烙印,继而写下理性女性裸体之美与现实主义精神完美结合所带来的无法形容的审美情趣,让广大市民阶层对人性美中所蕴含的生动与鲜活产生心灵的共鸣。

\subsection{历史层次}
欧洲艺术本来就是有描绘裸体女性艺术的历史传统的。
古希腊``神人同形同性''的宗教观念和审美观念早已根深蒂固于欧洲文化的内源。他们认为,神和人是相同的,都有着健美的身体与七情六欲。在这一时期,公民也就以健硕、健康、富于生命力为美(这恰巧也暗合了帅先生在课程绪论中讲到的``美的本质是旺盛的生命力''的著名论断);这一时期的裸体绘画、雕塑大量出现,虽然以男性裸体居多,但也不乏优秀的裸女艺术作品,如普拉克西特列斯的着力刻画丰满圆浑的女性肢体的《克尼多斯维纳斯》,如即使毁损但风韵不减的《米洛斯的阿芙洛狄忒》,轻飘的裙子滑落腰际。这些雕刻细腻、线条生动的大理石塑像,表现了当时人们对澎湃生命力的赞美。

后来,随着基督教的兴起,这种热情赞美裸女魅力的激情被生硬地压制了下去。人们受意识形态的影响,不再敢于表现天然的健康的躯体。但是,这种反人性的压制终究是不可能是长久的;在中世纪的漫长黑暗岁月里,即使是在修道院的隐晦角落里,仍不乏男欢女爱的场景。所以,当中世纪逐渐没落,传扬``人文主义思潮''的文艺复兴作为一个崭新的阶段登上历史舞台时,这一种通过描绘裸体女性胴体来抒发对强健、优美生命力的赞美的艺术形式必然是会重新赢得艺术家们的青睐,再次耀光辉于当世的。

就像著名学者,画家陈醉说道的那样,``人类在把握裸露的同时,却更深刻地认识到掩蔽的意义······''压制人们对艺术真诚的追求,就``取消了的是人类的创造本能''!\footnote{陈醉,《裸体艺术百年风云录》,中央艺术报,2011,3,21}

\subsection{社会层次}
对女性身体的崇拜是自母系社会早期就已经诞生的。当时,女性的身体,尤其是其具有的生殖力,被很多种族奉为崇高、万能、不可侵犯的对象。

后来,随着父系社会的崛起,以及在狩猎中男性的力量、速度等要素是不可缺少的,于是,男性在社会中的地位越来越高,女性的地位也就逐渐变低,在某些历史时期的某些地方,女性甚至成为了男性的附庸,受到了非人的待遇。例如,古希腊人将裸体视为神圣,并且认为只有男性躯体才具有神性,而女性更多的是被认为是代表本能和爱欲的低俗身体。

这种违反人性的做法势必是不可长久的。在文艺复兴时期,随着对中世纪黑暗枷锁的挣脱与人文主义的传播,女性的外表美得到了社会的认同与赞美。女性的身体开始得到了解放,女性的地位也得到了空前的提升。之前对女性身体的``遮蔽''的规训不见了;女性可以自由的彰显自己躯体的优美曲线与生命的蓬勃力量。于是,这一时期,在装饰方面,女性的服饰的裸露度开始增加;在艺术方面,女性艺术也呈现裸体化的倾向。

总结来说,这种女性裸体美的逐渐彰显是与社会意识的发展密切联系的。就像帅先生在课程开篇提到的``审美活动中的两个重要发现''中提出的那样:``美的形态随着社会形态的发展而发展,社会形态是一个新的社会形态对旧的社会形态的不断扬弃过程,因此,美的形态也是一个新的美的形态对旧的美的形态的不断扬弃过程。''在复兴古典艺术的过程中,女性裸体艺术正是在对旧的女性着装``遮蔽''艺术的不断扬弃中发展起来的,而这,是与时代的人文主义新旋律是分不开的。这从某一个侧面,也可以反映出女性社会地位的抬升。

\subsection{审美层次}

世俗的审美观在文艺复兴时期经历了一个翻天覆地的大变革。

中世纪基督教会推行严酷的禁欲主义``摧残人性'',人们失去了思想自由和幸福生活的权利,``连哭与笑都成为触犯上帝的罪行'',所以在中世纪的画像中包括圣母在内都是呆板僵冷且毫无感情'的,这也是这个时期人们审美观念的集中表现:严肃,端庄,不苟言笑。在他们的眼中,世俗的东西都是可鄙的,卑劣的,真正的幸福在于天堂。但是,他们从来也没有想到,天堂是一个他们永远也接触不到的、人造的幻象;把自己的幻想结成玩偶,然后再将自己的全部信念、全部依靠都投放在其上,这在我看来,是一件有些悲哀的事,对于那些不曾有坚定的宗教信仰的人来说,尤其可悲。

但在文艺复兴时期,上述的观念受到了强烈的冲击,并最终被打碎、碾成齑粉。

文艺复兴是一个神走下神坛,而人性的光辉开始登上宝座并照耀世间的时期。把人作为宇宙的中心,高度赞扬人的理性与精神,是这一时代的主旋律;其在市民阶层的审美层次的表现,就是``以肉欲为中心的人体美崇拜'';这里的``肉欲''不能简单地等同于性欲冲动,而是一种对单纯天然、自由开放、毫无服饰束缚的人体美的赞美与渴望。裸体艺术在文艺复兴时期被视为一种新的时尚,一种对美的追求与崇拜。相比于过去的严肃的审美观念,这种对由裸体艺术所展现出来的自由奔放的生命力的欣赏,有力地表现了人们对世俗生活的享受与青睐。人们不再着迷于死后,而是开始了真真切切的生活,活在当下。

``每个时代,每个社会都将其本质晶化于观念形态乃至于一切精神发现。时代和社会在意识形态之上,将自己表现于哲学、科学、法制、文学、艺术、行为准则以及人体美观念,宣扬一定的美的准则从而设计出时代和社会的理想形式。''\footnote{刘华英,《意大利文艺复兴名画中的妇女形象》,山东艺术学院学报,2008,6}在文艺复兴时期,女性裸体艺术是人文主义的开放、自由精神在艺术领域的集中体现,也是人文主义者的一件极其有力的武器;正是它借助了人们巨大的原始冲动——``力比多'',方成功地将人们的注意力由天堂引向人间。从此,人们不再狂热地迷恋着死后,而是热切地关注着当下。


\section{总结}
正如法国哲学家丹纳曾在《艺术哲学》中描述到的,``人们摆脱了禁欲主义与教会的统治,懂得关心自然界,健康、美和快乐。中世纪的精神到处在变质、瓦解。''在文艺复兴这场历史大变革中,人性的力量战胜了神灵的力量,世俗的、鲜活的``美''战胜了宗教的死板的教条。而其中,女性裸体,被看作是灵与肉的和谐统一,作为反对宗教禁欲主义的最强有力的武器,被人文艺术家们广泛应用;他们对女性裸体美的直率而坦诚的表达,让人们得到了审美的享受的同时,也接受了资本主义自由、开放的新时尚和充裕、富足的资本主义新生活。从此,历史的篇章翻开了新的一页,在欧罗巴的广阔土地上,中世纪的阴霾已经烟消云散,资本主义的新光辉从此开始照耀升腾。


\bibliographystyle{IEEEtran}
\bibliography{reference}


	帅松林,
	 \emph{审美的历程}.
	清华大学出版社 ,
	2,2014.\\


	
丹纳,
	 \emph{艺术哲学}.
北京人民美术出版社,
1963\\

	
	恩格斯,
	\emph{路德维希·费尔巴哈和德国古典哲学的终结}.
	人民出版社\\




	陈醉,
	\emph{裸体艺术百年风云录}.
	中央艺术报,
	3,2011\\


	兰岗,
	\emph{主题与变奏——中国仕女图与西方女性裸体画的比较}.
	贵州大学艺术报,
	4,2005\\



	刘华英,
	\emph{意大利文艺复兴名画中的妇女形象}.
	山东艺术学院学报,
	6,2008\\



	程成,
	\emph{论女性裸体形象在西方绘画中广泛存在的原因}.
	书画长廊,
	4,2012\\

	覃志峰,
	\emph{论裸体艺术与希腊神话}.
	艺术探索,
	10,2006



\end{document}
